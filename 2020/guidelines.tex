%%%%%%%%%%%%%%%%%%%%%%%%%%%%%%%%%%%%%%%%%%%%%%%%%%%%%%%%%
%%   $RCSfile: example.tex,v $
%%  $Revision: 1.1 $
%%      $Date: 2004/08/11 09:52:43 $
%%     Author: Stefan Mueller (CL Uni Bremen)
%%    Purpose: Example file for contributions to the HPSG proceedings
%%   Language: LaTeX
%%%%%%%%%%%%%%%%%%%%%%%%%%%%%%%%%%%%%%%%%%%%%%%%%%%%%%%%%


\documentclass[11pt,a4paper,fleqn]{article}

% use xelatex

\usepackage{fontspec}

\usepackage{times,float}


\usepackage[sectionbib]{natbib}

\usepackage{german}
\selectlanguage{USenglish}

% If you want to make your paper more usable for readers,
% insert the following code.
% It links references to sections, examples, and citations
% in the document and provides bookmarks for sections and subsections.

\usepackage[bookmarks=true,bookmarksopen=true,hyperindex=true,breaklinks=true,draft=false,plainpages=false,hyperfootnotes=false,%
colorlinks=false, pdfborder={0 0 0}%
]{hyperref}  



\title{Guidelines for Type Setting Your Contribution to the HPSG Proceedings}

\author{Stefan Müller\\
Humboldt Universität zu Berlin}

\begin{document}
\maketitle


\section{Things to Submit}

Please submit:
\begin{enumerate}
\item your paper in pdf format\footnote{
As with last year, we will produce a large pdf file that includes all
of the papers.  This way, readers can download the entire proceedings
as a book.

If you absolutely cannot do PDF, please let me know as soon as
possible and I will accomodate you in some way: point you to some
reasonable converters, help you with the conversion, etc.}
%
 named `kruger.pdf', if your lastname is \emph{Kruger}.



\item an abstract
\end{enumerate}
The abstract will be presented at a separate HTML page. You may submit it in ASCII [please remove \LaTeX{} mark up, thanks]
or---if you have special formatting---in HTML. 
To get an idea about how this looks like,
you may look at the proceedings of previous years at:\newline
\url{http://cslipublications.stanford.edu/HPSG/}\section{Deadline}

The deadline for submission is October 15th.

\section{Page Limit}

The page limit for full papers is 20 pages.

%The page limit for posters is 15 pages.

\section{Title Page}


The title page will be created automatically in order to be recognizable by google scholar.

\section{Abstract}

We encourage the inclusion of a short (one paragraph)  abstract directly on the first page.
(no pagebreak before the main text, please)


\section{Details on Formatting}

\subsection{\LaTeX{} Users}

\subsubsection{Styles}

Please adapt the \verb+example.tex+ file at:\newline
\url{http://cslipublications.stanford.edu/HPSG/2017/call.shtml}\newline
and use the BibTeX file provided there.

\subsubsection{PDF Creation}

If you do not fancy stuff that needs PostScript you may produce
the pdf file by using \verb+pdflatex+. If you use \verb+pstricks+ or
similar packages, please use \verb+dvips+ and \verb+ps2pdf13+.

Using \verb+ps2pdf13+ helps avoiding problems with ligaturs like in `fi'
or `ff'. If you do not have this on your machine, using \verb+ps2pdf+ is okay.

Please use the \verb+ps2pdf+-option \verb+-dAutoRotatePages=/None+  to avoid auto rotation
of figures.


\subsubsection{A4 Paper}

Page numbers are inserted automatically. For this process to work properly it is important to use
a4paper rather than letter. Most tex distributions in the US are set up to use letter paper as default.

If you are using \verb+dvips+, please use the option \verb+-t a4+. This tells
\verb+dvips+ to use A4 paper.

On a4 paper in general (dvips and pdflatex) see \url{http://www-h.eng.cam.ac.uk/help/tpl/textprocessing/latex2pdfprint.html}.

If you are a Mac user and use MacTex you can change this setting with the Tex Live Utility. Go to
Configure$|$Change Paper Size and select A4.

\subsubsection{Huge AVMs and Trees}
\label{huge}

Please typeset trees, avms, and formulae in a way that they fit the
\verb+\textwidth+. If you see no other way to reduce the size of the respective
objects, you may use \verb+resizebox+ from the package \verb+graphicx+.
{\small
\begin{verbatim}
\resizebox{\linewidth}{!}{%
\ms{
 synsem & \onems{ loc$|$cat$|$subcat del(\ibox{2},\ibox{1})\\
                }\\
 head-dtr & \onems{ synsem$|$loc$|$cat$|$subcat \ibox{1} \\
                  }\\
 non-head-dtrs & \sliste{\onems{ synsem \ibox{2}\\ 
                               }}\\[2mm]
}}
\end{verbatim}
}


\subsubsection{Hyphenation}
\label{hyphenation}

If you write things like \verb+head-driven+ or very long pathes like
{\sc snysem$|$loc$|$cat$|$head$|$mod$|$loc}, \LaTeX{} does not do hyphenation
(in the part following the dash).

If you use \verb+german.sty+ you get additional markup that allows for proper hyphenation:
\begin{verbatim}
head"=driven

{\sc snysem$|$""loc$|$""cat$|$""head$|$""mod$|$""loc}
\end{verbatim}
With this markup even long pathes like {\sc snysem$|$loc$|$cat$|$head$|$mod$|$loc$|$cat$|$""head}
are typeset properly. Alternatively you my write
\begin{verbatim}
{\sc snysem$|$\-loc$|$\-cat$|$\-head$|$\-mod}
\end{verbatim}
which introduces a dash at the place of the linebreak:
{\sc snysem$|$\-loc$|$\-cat$|$\-head$|$\-mod$|$\-loc$|$\-cat$|$\-head}.

If you use \verb+german.sty+ do not forget to declare English as the language
you are using:
\begin{verbatim}
\selectlanguage{USenglish}
\end{verbatim}
Otherwise the section name for references comes out in German.

\subsubsection{URLs}

There is a cool package: url.sty It helps you typeset URLs and together with the hyperref style the
URLs are even clickable. The `\~{}' symbol is correctly displayed with url.sty. Please use the
hyphens option for the url package. See example.tex for details.





\subsection{Others (StarOffice, Word Perfect, Open Office, Word, \ldots)}

\subsubsection{Font}

Please use a 11pt times (Type 1 font). 

\subsubsection{No Page Numbers}

Please submit your paper without page numbering.  I will add in
the page numbering on the PDF file as I am putting the
proceedings together.

\subsubsection{Justification}

Please make sure that the text is typeset in justification,
i.e. with the text aligned at both the left-hand side and the right-hand side.

\subsubsection{Papersize and Margins}
\label{margins}

Please use A4 paper:
\begin{table}[H]
\begin{tabular}{@{}ll}
paperheight  = 297mm\\
paperwidth   = 210mm
\end{tabular}
\end{table}
%
The text size should be:
\begin{table}[H]
\begin{tabular}{@{}ll}
textwidth  = 120mm\\
textheight = 201mm
\end{tabular}
\end{table}

\begin{table}[H]
\begin{tabular}{@{}l@{ = }l}
distance from the top & 50 mm\\
distance from the left-hand side & 45 mm\\
distance from the right-hand side & 44 mm\\
distance from the bottom & 48 mm\\
\end{tabular}
\end{table}

This is important since otherwise automatic page numbering will not work.
If you are not sure whether you did things right, you may compare your file
with (my) papers from the past years:\newline 
\url{http://cslipublications.stanford.edu/HPSG/5/toc.shtml#stmue}

\subsection{PDF Creation}
\label{sec-pdf-creation-word}

Please use LibreOffice to create the PDF. PDFs created by word will not be accepted, since AVMs do
not display correctly accross platforms.

\section{Things You Should Not Do}

\begin{itemize}
\item Do not change the margins or other \LaTeX{} internal values! Page numbers will be inserted automatically and
      if you changed things this will not work and cause me and you additional work.
\item Do not change the font size.
\end{itemize}


\section{Checklist}

\begin{itemize}
\item Do you have the right margins (\LaTeX{} $\to$ no problem if you didn't change
      the styles, others see Section~\ref{margins})?
\item Do you use a4 paper, rather than letter? dvips has to be used with the option ``-t a4'' if you
  work in a setup that has letter as the default.
\item Are you within the page limit of 20 pages?
\item Did you suppress the page numbers?
\item If you look at figures in the pdf file, do rotated figures flip back?
      Please edit your text in a way that figures do not flip back, since this
      breaks the process of automatic page numbering.

If you used \LaTeX{} to produce your document you can avoid rotating figures by calling
\verb+ps2pdf13+ with the option \verb+-dAutoRotatePages=/None+.
\item Is there something sticking out at the right-hand side of the text?
      If so, see Section~\ref{hyphenation}.
\item Are there coloured URLs or something that like in your paper? If so,
      please change this to the text colour.
\item Are the URLs underlined? If so please change this to normal text.
\item Do you have AVMs, equations, or trees that do not fit the textwidth?
      If so, please fix this. See Section~\ref{huge}.
\item Is the name of the pdf file `author.pdf', where `author' is your lastname or a list of the lastnames of several authors
      separated by `-'?
\item Does the paper contain an abstract?
\item Do you have the abstract ready to send separately? (please put plain text directly into the mail, the
format of the abstract is given on the next page)
\item Did you remove all \LaTeX{} markup from the abstract? (Yes, `\verb+---+' is also markup)
\item If you are a Word user: Did you use LibreOffice for PDF creation? See Section~\ref{sec-pdf-creation-word}.
\end{itemize}

Please do not send zip files
or word files or anything else. Just the abstract/author information in the mail and one pdf-attachment.



Great! I am looking forward to your submission!
Please send the pdf and the abstract/author information to\newline
\href{mailto:St.Mueller@hu-berlin.de}{St.Mueller@hu-berlin.de}. 


If you have any questions, please contact me at the given adress.

\newpage

\noindent
The abstract/author information has to be part of the email and looks as follows:

First Author \& Second Author \# First Affiliation \& Second Affiliation \# Titel of Paper \# Date of
Submission \# \# Abstract \# Keywords

\bigskip

Example:

\medskip
\noindent
Song, Sanghoun \& Bender, Emily M. \# University of Washington \& University of Washington \#
Individual Constraints for Information Structure \# October 16, 2012 \# \#
This paper, in the context of multilingual MT, proposes the use of ICONS
(Individual CONstraintS) to add a representation of information structure to
MRS. The value of ICONS is a list of objects of type info-str, each of which
has the features CLAUSE and TARGET. The subtypes of info-str indicate
which information structural role is played by the TARGET with respect
to the CLAUSE. This proposal is designed to support both the calculation
of focus projection from underspecified representations and the handling of
multiclausal sentences. \# HPSG, Information Structure

\medskip



%\bibliographystyle{natbib.fullname} 
%\bibliography{biblio}

\end{document}


