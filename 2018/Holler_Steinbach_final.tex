%%%%%%%%%%%%%%%%%%%%%%%%%%%%%%%%%%%%%%%%%%%%%%%%%%%%%%%%%
%%   $RCSfile: example.tex,v $
%%  $Revision: 1.1 $
%%      $Date: 2004/08/11 09:52:43 $
%%     Author: Stefan Mueller (CL Uni Bremen)
%%    Purpose: Example file for contributions to the HPSG proceedings
%%   Language: LaTeX
%%%%%%%%%%%%%%%%%%%%%%%%%%%%%%%%%%%%%%%%%%%%%%%%%%%%%%%%%

% use this to check the margins
%\documentclass[11pt,a4paper,fleqn,draft]{article}

% use the following once there are no problems with the margins
\documentclass[11pt,a4paper,fleqn]{article}


%\usepackage[latin1,utf8x]{inputenc} % unicode if you do not use xelatex
\usepackage{fontspec} % using xelatex is highly recommended

%\usepackage{times} % veraltet und nicht kompatibel mit Kapitälchen in manchen Umgebungen
%\usepackage{mathptmx} % nicht kompatibel mit Kapitälchen in manchen Umgebungen
\usepackage[scaled=0.90]{helvet}
\usepackage{courier}

\pagestyle{empty}

\usepackage[sectionbib]{natbib}

\usepackage{polyglossia}
	\setmainlanguage[variant=american]{english}
	\setotherlanguage[spelling=new]{german}

\usepackage{my-gb4e-article}
\usepackage{csquotes}
%\usepackage{styles/avm}
\usepackage{merkmalstruktur}
\usepackage{tabularx}
%\usepackage{parskip} %sonst werden alle AVMs eingerückt
\usepackage{qtree}
%\usepackage[linguistics]{forest}


% german.sty has nice hyphenation support
%\usepackage{german}
%\selectlanguage{USenglish}

% this package supports scaling for instance if your AVMs do not fit the page
\usepackage{graphicx}
%\resizebox{\linewidth}{!}{
%\ms{
%      synsem & \onems{ loc$|$cat$|$subcat {\rm del(\ibox{2},\ibox{1})}\\
%                     }\\
%      head-dtr & \onems{ synsem$|$loc$|$cat$|$subcat \ibox{1} \\
%                       }\\
%      non-head-dtrs & \sliste{\onems{ synsem \ibox{2}  \\ }}\\[2mm]
%}}

% includes page numbers for references
%\usepackage{varioref}
% If you want to make your paper more usable for readers,
% insert the following code.
% It links references to sections, examples, and citations
% in the document and provides bookmarks for sections and subsections.

% For URLs load the style file url.sty , You may then type \url{http://hpsg.fu-berlin.de/~stefan/}
% without any hassle regarding the `~'. 
\usepackage[hyphens]{url}
\urlstyle{same}

\usepackage[bookmarks=true,bookmarksopen=true,hyperindex=true,breaklinks=true,draft=false,plainpages=false,hyperfootnotes=false,%
colorlinks=false, pdfborder={0 0 0}%
%,pdftex=true  % use this if you use pdflatex
%,ps2pdf=true  % use this if you use ps2pdf, i.e. if you use tree-dvips
]{hyperref}
\hypersetup{colorlinks=false, pdfborder={0 0 0}}

\title{Sign language agreement: A constraint-based perspective}
\author{Anke Holler and Markus Steinbach \\ University of Göttingen}

\begin{document}
% Title is generated automatically.
%\maketitle
\begin{abstract}The paper addresses verbal agreement in German Sign Language from a constraint-based perspective. Based on Meir's Agreement Morphology Principles it presents an HPSG analysis of plain, regular and backwards agreement verbs that models the interaction between phonological (manual) features and syntactico-semantic relationships within a verbal sign by well-defined lexical restrictions. We argue that a sign-based declarative analysis can provide an elegant approach to agreement in sign language since it allows to exploit cross-modular constraints within grammar, and hence permits a direct manipulation of all relevant phonological features of a verb depending on its syntactic and semantic properties.
\end{abstract}

\setcounter{footnote}{2}
\renewcommand{\thefootnote}{\fnsymbol{footnote}}
\footnotetext{
We thank the reviewers and the audience of the HPSG 2018 conference in Tokyo for discussion and valuable comments.}
\renewcommand{\thefootnote}{\arabic{footnote}}
\setcounter{footnote}{0}


\section{Introduction}

Agreement between a verb and two of its arguments is one of the best studied areas in sign language linguistics (Lillo-Martin \& Meier 2011). The range of analyses varies from gesturally oriented approaches via semantic, i.e. thematic, accounts up to purely syntactic implementations.\footnote{For a deeper discussion of these analyses, see Salzmann et al. 2018.} In the present paper, we argue for a constraint-based modeling of sign language agreement because it allows for a combination of the insights of both semantic and syntactic approaches. As we will show below, a constraint-based account has the noteworthy advantage that (manual) phonological features of verbs that inflect for agreement, such as beginning and end point of path movement as well as hand orientation, can be manipulated in a direct way.


\begin{figure}[h]
    \centering
	\includegraphics[width=0.45\textwidth]{figuresigningspace.jpg}
    \caption{Signing space} \label{fig:schwan1}
\end{figure}

Agreement in sign languages is locus agreement, which means that it is expressed in the signing space by a manipulation of phonological features. The relevant phonological features of the verb agree with or depend on the referential locations (R-loci) the discourse referents of the subject and object are linked to in the signing space (Steinbach \& Onea 2016). These R-loci are either actual locations of present referents (i.e. deictic locations) or locations that are assigned for non-present referents on the horizontal plane of the signing space (i.e. anaphoric locations, cf. figure 1). Non-present discourse referents can be localized in various ways. One major strategy is the use of determiner-like signs such \textsc{index}\textsubscript{X} and \textsc{poss}\textsubscript{X}. The first referent is typically assigned to the ipsilateral area of the signing space and the second one to the contralateral area (cf. figure 2).

\begin{figure}[h]
    \centering
	\includegraphics[width=0.45\textwidth]{figurerefs.jpg}
    \caption{Localization of referents} \label{fig:schwan2}
\end{figure}

To give an example: In the first sentence of (\ref{ex:teacher}) below, the first discourse referent \textsc{maria} is localized with the pointing sign \textsc{index}\textsubscript{3a} at location 3a, which is the ipsilateral area of the signing space, i.e. the right side for right-handed signers (cf. figure 2). Similarly, the second discourse referent \textsc{new teacher} is localized at the contralateral area of signing space, i.e. 3b. This R-locus is then used to pronominalize \textsc{new teacher} in the second sentence.\footnote{This is a second revised version of the originally published paper because of an obvious erratum in referencing to example (\ref{ex:teacher}). There are no other differences to the first version instead of replacing this erroneous paragraph [date of correction: 15-10-2019].}

\begin{exe}
	\label{ex:teacher}
	\ex \textsc{m-a-r-i-a index}\textsubscript{3a} \textsc{teacher new index}\textsubscript{3b} \textsc{like. index}\textsubscript{3b} \textsc{smart.}\\
	‘Maria likes the new teacher. S/he is smart.’
\end{exe}

The two R-loci introduced in the first sentence can also be used to express agreement on the verb \textsc{give} by moving from the R-locus associated with the subject to the R-locus associated with the object. This is illustrated by (2) below. Hence, sign languages, just like spoken languages, use similar means for pronominalization and agreement. However, unlike spoken languages, sign languages do not use sequential agreement affixes but express referential indices of the subject and object simultaneously on the verb. (cf. Aronoff et al. 2005). 

\begin{exe}
	\ex \label{ex:poss}
	\textsc{yesterday} \textsc{index}\textsubscript{3a} \textsc{book} \textsubscript{3a}\textsc{give}\textsubscript{3b}\\
	‘Yesterday she gave him a book.’
\end{exe}

We address verbal agreement in German Sign Language (DGS) from a constraint-based perspective in this paper. In particular we aim at modeling well-known restrictions on agreement in sign languages in such a way that the interaction between phonological (manual) features and syntactico-semantic relationships can be adequately described. We show that a constraint-based approach offers an elegant analysis of sign language agreement since it permits a direct manipulation of the relevant phonological features of the verb sign. 

This article is organized as follows: In the following section, we describe the general properties of agreement in sign languages and introduce the three most important verb classes, i.e. regular agreement verbs, backwards agreement verbs and plain verbs. Section 3 introduces so-called agreement auxiliaries that are used to mark agreement manually in the case pf plain verbs which are not able to express agreement. In section 4 we present and discuss a constraint-based analysis of sign language agreement couched in standard HSPG theory, that is, our analysis does not depend on any modality-specific assumptions or modifications of HPSG.


\section{Agreement in sign languages}

The huge amount of studies on agreement in many different sign languages has shown that agreement in the visual-gestural modality differs in several respects from agreement in the auditory-oral modality (Lillo-Martin \& Meier 2011; Mathur \& Rathmann 2012; Salzmann et al. 2018). 

First of all, it is well documented that not all verbs in a sign language are able to realize verbal agreement overtly. In addition to so-called agreement verbs such as \textsc{give} in example (2) above, sign languages also have so-called plain verbs such as \textsc{like} in the first sentence of example (1) above, which cannot be inflected for agreement. A third class of verbs are so-called spatial verbs, whose beginning and endpoints are not determined by arguments of the verb (or grammatical functions) but by topographic referents. Like agreement verbs, spatial verbs can be spatially modified but the controller of the agreement is not a locus with a referential interpretation but a locus with a topographic interpretation (e.g. the village on the left, the house on the right, $\ldots$). Examples of typical DGS verbs for each of these three verb classes are listed in (\ref{ex:verbclasses}). In the following, we ignore spatial verbs since the topographic relations expressed are not agreement relations in the strict sense but decriptions of the location or movement of an entity in the real world. 

\begin{exe}
\ex 	\label{ex:verbclasses}
\begin{xlist}
\ex	Agreement verbs: \textsc{give, help, teach, ask, visit, show,} …	
\ex Plain verbs: \textsc{like, know, wait, think, buy,} …
\ex Spatial verbs: \textsc{move, put, stand, lie, be-at,}  …
\end{xlist}
\end{exe}

Secondly, verbs in sign languages express agreement with their arguments directly in the signing space by path movement and/or orientation of the hands (i.e. palm orientation or orientation of the fingertips, cf. Meir 1998). With the DGS verb \textsc{give} in (\ref{ex:loc1}), path movement begins at x, the R-locus associated with the discourse referent of the subject, and ends at y, the R-locus associated with the discourse referent associated with the object. By contrast, the DGS verb \textsc{influence} does not only express agreement by path movement but also by orientation of the hands. In (\ref{ex:loc2}) the fingertips are oriented towards the location associated with the object, i.e. y.  

\begin{exe}
	\ex \label{ex:loc}
	\begin{xlist}	
	\ex \textsubscript{X}\textsc{give}\textsubscript{Y} 	\\
		‘to give something to someone’	\label{ex:loc1}
	\ex \textsubscript{X}\textsc{influence}\textsubscript{Y}\\
		‘to influence someone’			\label{ex:loc2}
	\end{xlist}
\end{exe}
	
The examples in (\ref{ex:loc}) also illustrate another property of agreement in sign languages: Verbs in DGS do not only agree with the subject (first argument) but also with the object (second argument). Subject and object agreement is the standard case not only in DGS but also in many different unrelated sign languages.

A fourth important property of agreement in sign languages is that it affects directly the phonological form of the verb. Agreement is expressed through the manipulation or specification of the two phonological features hand orientation and path movement of the corresponding agreement verb. Consequently, phonological properties of the verb may block the overt realization of agreement. This is the case with plain verbs: Agreement with subject and object is prohibited because hand orientation and the beginning and endpoint of path movement are lexically specified. Consider, for instance, the plain verb \textsc{like} in example (\ref{ex:teacher}) above. Path movement always involves a downward movement of the dominant hand in front of the signers chest. Therefore, this movement cannot be modified and adapted to the R-loci that subject and object are linked to. Even with agreement verbs, agreement may sometimes be blocked by phonological constraints. In some varieties of DGS, verbs like \textsc{trust} only agree with first person subjects and non-first person objects because the beginning of the path movement is lexically specified (i.e.~the forehead of the signer). In other varieties of DGS, however, the verb \textsc{trust} also inflects with non-first person subjects and first person objects, which means that it has already been developed into a full subject-object agreement verb. In these varieties, the inflected form in (\ref{ex:trust2}) would be grammatical.

\begin{exe}
	\ex \label{ex:trust}
	\begin{xlist}	
	\ex	\textsubscript{1}\textsc{trust}\textsubscript{2}		\\
		‘I trust you.’	\label{ex:trust1}
	\ex  *\textsubscript{2}\textsc{trust}\textsubscript{1}	\\
		‘You trust me.’ \label{ex:trust2}
	\end{xlist}
\end{exe}

A fifth unique property of sign language agreement, which is highly relevant for each analysis, is the distinction between two different kinds of agreement verbs: regular and backwards agreement verbs. Regular agreement verbs follow the pattern described above. The path movement starts at the R-locus associated with the subject and ends at the R-locus of the object. By contrast, backwards agreement verbs such as \textsc{invite} in (\ref{ex:inviteyou}) below show the reverse pattern. The path movement begins at the position of the object and ends at the position of the subject. Interestingly, the hand is always oriented towards the object, even with backwards agreement verbs. We will see that the general distinction between regular and backwards agreement verbs (i.e. the difference in movement direction) can be derived from thematic restrictions discussed in Meir (1998, 2002). By contrast, the specification of the hand orientation follows from syntactic restrictions on the \textsc{comps} list.

\begin{exe}
\ex \textsubscript{2}\textsc{invite}\textsubscript{1}\\
	‘I invite you.’ \label{ex:inviteyou}
\end{exe}

The following figure 3 gives an overview of the agreement picture described in this section. Note that these modality-specific properties of agreement in sign languages and the specific verb classes follow from the spatial nature (path movement and orientation) of sign language agreement and its gestural origins (transfer of an entity). This does, however, not mean that sign language agreement is not part of the linguistic system (for a more detailed discussion, cf. Salzmann et al. 2018).  

\begin{figure}[h]
\centering
\Tree
[.Verbs
	\qroof{\textsc{like, know, wait, think}}.{Plain Verbs}
	[.{Agreement Verbs}
		\qroof{\textsc{\textsubscript{x}help\textsubscript{y}, \textsubscript{x}give\textsubscript{y}}}.{Regular\\Agreement Verbs} 
		\qroof{\textsc{\textsubscript{y}invite\textsubscript{x}, \textsubscript{y}pick up\textsubscript{x}}}.{Backwards\\Agreement Verbs}
	]
]
\caption{Verb classes in German Sign Language}\label{verbclasses-tree}
\end{figure}


\section{Agreement auxiliaries}

In the previous section, we have shown that plain verbs such as \textsc{like} in example (\ref{ex:teacher}) cannot be inflected for agreement. Interestingly, many sign languages have developed various grammatical means to overcome the agreement gap caused by plain verbs. These sign languages make either use of a specific class of auxiliaries (so-called agreement auxiliaries) or they use non-manual markers such as eye gaze and head tilt to express the agreement relations with plain verbs (Steinbach \& Pfau 2007; Sapountzaki 2012; Neidle et al. 2000; Thompson et al. 2006). 

In the following, we only focus on agreement auxiliaries since DGS belongs to the group of sign languages that make use of manual agreement markers. Like agreement verbs, agreement auxiliaries express subject and object agreement by means of path movement and hand orientation. Agreement auxiliaries in sign languages differ from typical spoken language auxiliaries in that they are not used to mark tense, aspect, modality, or voice (so-called TAM auxiliaries) but ‘‘only'' to mark agreement with the subject and the object. Genuine agreement auxiliaries seem to be rare in spoken languages. The German auxiliary \textit{tun} (‘to do’) in (\ref{ex:tun1}), which is frequently used in colloquial variants of German and in many German dialects, might be an exception to this generalization. Unlike other auxiliaries in German, \textit{tun} is not a TAM marker, it is not restricted to certain semantic contexts (the corresponding sentence without \textit{tun} in example (\ref{ex:tun2}) is functionally identical to its counterpart in (\ref{ex:tun1})) and its use seems to be functionally very similar to agreement auxiliaries in sign languages (Erb 2001; Steinbach \& Pfau 2007).

\begin{exe}
\ex \label{ex:tun}
\begin{xlist}
\ex	\label{ex:tun1}
	\gll Sie tu-t ein Buch les-en.\\
	She do-3.sg a book read-inf\\
\ex	\label{ex:tun2}
	\gll Sie lies-t ein Buch.\\
	She read-3.sg a book\\
	‘She is reading a book.'
\end{xlist}

\end{exe}

The auxiliary \textit{tun} seems to be some kind of dummy auxiliary that is only used to express morphosyntactic features such as present and past tense and agreement. Note that these features can always be optionally expressed by the main verb as illustrated in (\ref{ex:tun2}). Hence, \textit{tun} resembles the DGS agreement auxiliary, the Person Agreement Marker \textsc{pam} (cf. Rathmann 2003; Steinbach \& Pfau 2007).

The source of the DGS agreement auxiliary \textsc{pam} is the noun \textsc{person} as demonstrated by figure 4. Contrary to \textsc{pam}, \textsc{person} does not exhibit a directional movement but only a simple downward movement.  The agreement auxiliary \textsc{pam}, however, expresses the agreement relation by a manipulation of the phonological features path movement and hand orientation and behaves in this respect just like regular agreement verbs. 



\begin{figure}[h]
    \centering
	\includegraphics[width=0.45\textwidth]{figurepam.jpg}
    \caption{From noun to auxiliary in DGS} \label{fig:schwan4}
\end{figure}


\textsc{pam} can be used with plain verbs as in (\ref{ex:pam1}), with adjectival predicates as in (\ref{ex:pam2}), and with verbs like \textsc{trust}, which cannot be inflected for non-first person subject agreement and first person object agreement as in (\ref{ex:pam3}).


\begin{exe}
\ex  \label{ex:pam}
\begin{xlist}	
\ex \textsc{mother}  \textsc{index}\textsubscript{3a}  \textsc{neighbor new}  \textsc{index}\textsubscript{3b} \textsc{like}   \textsubscript{3a}\textsc{pam}\textsubscript{3b}	\\
		‘(My) mother likes the new neighbor.’ \label{ex:pam1}
	\ex	\textsc{index}\textsubscript{1}  \textsc{poss}\textsubscript{1}  \textsc{brother}  \textsc{index}\textsubscript{3a}  \textsc{proud}   \textsubscript{1}\textsc{pam}\textsubscript{3a}\\
		‘I am proud of my brother.’ \label{ex:pam2}
	\ex	\textsc{index}\textsubscript{2}  \textsc{trust}   \textsubscript{2}\textsc{pam}\textsubscript{1}\\
		‘You trust me.’	\label{ex:pam3}
\end{xlist}
\end{exe}

%\begin{figure}[h]
%\centering
%\includegraphics[width=0.35\textwidth]{person.pdf}
%\caption{From noun to auxiliary in DGS}\label{person}
%\end{figure}
%
%\begin{figure}[h]
%\centering
%\includegraphics[width=0.35\textwidth]{pam.pdf}
%\caption{From noun to auxiliary in DGS}\label{pam}
%\end{figure}

Note that there seems to be some variation in the syntactic position of \textsc{pam}. In Southern German variants, \textsc{pam} is preferably inserted in pre-verbal position (even before the object) as can be seen in (\ref{ex:hans}), whereas in most variants of DGS, \textsc{pam} is usually inserted in post-verbal position as illustrated in example in (\ref{ex:pam1}) above (Rathmann 2003; Macht 2016; Macht \& Steinbach, to appear). In example (\ref{ex:hans}), ‘\textsc{h-a-n-s}\textsubscript{3a}' means that the name ‘Hans' is fingerspelled at the location 3a, i.e. fingerspelled names can be directly linked to R-loci. 

\begin{exe}
\ex	 \textsc{h-a-n-s}\textsubscript{3a}  \textsubscript{3a}\textsc{pam}\textsubscript{3b}  \textsc{m-a-r-i-e}\textsubscript{3b}  \textsc{like} \\
‘Hans likes Maria.’\label{ex:hans}
\end{exe}

Interestingly, \textsc{pam} can also be combined with uninflected agreement verbs. Although this combination seems to be less acceptable than the version with inflected agreement verb without \textsc{pam}, it reveals interesting insignts in the inflectional pattern of \textsc{pam}. With uninflected backwards verbs like \textsc{invite} in (\ref{ex:invite}), \textsc{pam} does not follow the inflectional pattern of the backwards verb but moves from the position of the subject to the position of the object, i.e. even in the context of backwards agreement verbs, \textsc{pam} inflects like a regular agreement verb. Hence, the semantically empty agreement auxiliary \textsc{pam} generally expresses agreement with subject and object, no matter of the thematic structure of the corresponding main verb. 

\begin{exe}

\ex	\textsc{index}\textsubscript{3a}  \textsc{index}\textsubscript{3b}  \textsc{invite}  \textsubscript{3a}\textsc{pam}\textsubscript{3b} \\
		‘S/he invites him/her.’  \label{ex:invite}
\end{exe}

Consequently, \textsc{pam} is not subject to any semantic restriction and can be used with all kinds of plain verbs.

\begin{exe}
\ex	DGS plain verbs that express agreement by means of \textsc{pam}: \\
\textsc{be-proud, be-angry, know, like, trust, wait, be-interested-in, laugh}, …
\end{exe}

Note finally that \textsc{pam} can also be productively used to extend the argument structure of the main verb. Since \textsc{pam} is a transitive agreement marker, it can be used as a transitivizer in DGS.

\begin{exe}
\ex \label{ex:laugh}
\begin{xlist}	
\ex 		\textsc{index}\textsubscript{1}  \textsc{laugh}  \textsubscript{1}\textsc{pam}\textsubscript{2}	\\
‘I laugh at you.’
\ex	\textsc{index}\textsubscript{1}  \textsc{letter write}  \textsubscript{1}\textsc{pam}\textsubscript{2}\\
‘I write a letter to you.’
\end{xlist}	
\end{exe}


\section{A lexical analysis of agreement in HPSG}

The specific phonological and semantic properties of the three different verb classes discussed in the previous section and the interaction between their formal (phonological and syntactic) and semantic (thematic) properties call for a constraint-based lexical treatment of verbal agreement in sign language. Such an approach not only enables the formulation of cross-modular restrictions within grammar but also  allows for a direct relation of phonological and argument structural information within a sign. In particular, the thematic conditions and the interaction with phonological features highlighted in the previous section can explicitely be stated in the lexical entry of a verb. Such an approach perfectly meets with the insights formulated in the thematic approach in Meir (1998, 2002) and the HPSG account sketched in Cormier et al. (1999). In this section, we build on these two approaches and develop an HPSG analysis of (regular and backwards) agreement and plain verbs on the one hand and the agreement auxiliar \textsc{pam} on the other.


\subsection{Basic assumptions for lexical signs}

A lexical item in sign language structurally differs from a lexical sign in spoken language as it includes a description of the manual phonology, which consists of a particular handshape, a location, a movement, and a hand orientation as well as a description of the non-manual phonology (whose lexical aspects we mainly ignore  in the following). Thus, the phonological component of a sign language is much more complex than the corresponding phonological component of a spoken language. This has its reflex in the structure of the \textsc{phon} value. Following Safar \& Marshall (2004), we assume that \textsc{phon} represents relevant aspects of non-manual phonology such as the face, especially the brows, and the mouthing as well as comprises fine-grained information about the hand(s) with respect to shape, orientation and movement. A partial description of \textsc{phon} adapted from Safar \& Marshall (2004) is given in (\ref{ex:brow}). The most important part for our analysis of agreement is, of course, the manual features  movement and orientation. Movement of the hand(s) is defined by two positions in the signing space which mark the beginning and the end point of the movement. Orientation is defined by palm and finger orientation.
 

\begin{exe}
\ex  \label{ex:brow}
\resizebox{\linewidth}{!}{
\ms{
	phon& \ms{face|brow & brow\\
		manual & \ms{
			handshape & handshape \\
			orientation & \ms{
				palm & palm\\
				finger|index|loc & locus\\
			}\\
			movement & \liste{\ms{
					begin|index|loc & locus\\
					end|index|loc & locus\\
			}}\\
		}\\
		mouth|picture & picture\\
	}\\
}}
\end{exe}

As discussed in the previous sections discourse referents (and thus indices) in sign languages are linked to R-loci in the signing space. In order to represent these R-loci, we have to redefine the \textsc{index} value as is also illustrated in (\ref{ex:brow}). To account for the differences between spoken and sign languages with respect to their index values we suggest to define two new subtypes of the type \textit{index}, called \textit{categorial\_index}  and \textit{positional\_index} as is depicted in (\ref{ex:index}). This accounts for our general assumption that the type \textit{index} can be thought of as an HPSG analog of a reference marker in Discourse Representation Theory (cf. Kamp \& Reyle 1993).

\begin{exe}
\ex  \label{ex:index}
\begin{xlist}
\ex
\ms{
	index & \ms[categorial\_index]
	{person & person\\
		number & number\\
		gender & gender\\ 
	}\\
}

\ex
\ms{
	index & \ms[positional\_index]
	{locus & locus}\\
}
\end{xlist}
\end{exe}

The \textsc{index} value of type \textit{categorial\_index} is exploited for spoken languages and represents the usual morpho-syntactic features like person, number and gender. However, for sign languages, we follow Cormier et al. (1999) in stipulating a \textit{positional\_index} which refers to specific loci in the signing space. These are represented by a \textsc{locus} value. For the \textsc{locus} feature a type \textit{locus} is appropriate which is further partioned into the subtypes \textit{speaker}, \textit{addressee} and \textit{other}, where \textit{other} subsumes a set of variables, \textit{i}, \textit{j}, \textit{k}, etc., representing possible indices. 

Next, we come back to the observations concerning agreement in sign languages discussed in the previous section. We will develop an analysis that accounts for the two basic verb classes in DGS, i.e. agreement verbs as well as plain verbs. 


\subsection{Agreement verbs}
We follow Meir (1998, 2002) in the distinction between two kinds of agreement in sign language, (i) thematic agreement, and (ii) syntactic agreement, as formulated in the Agreement Morphology Principles (AMP). Below, we implement the AMP directly into our HPSG analysis to take up the generalization that thematic agreement marks the direction of the path movement (see \ref{ex:amp1}) whereas syntactic agreement is responsible for the orientation of the hand(s) (see \ref{ex:amp2}). 

\begin{exe}
\ex  Agreement Morphology Principles (AMPs):\label{ex:amp}
\begin{xlist}	

\ex The direction of the path movement of agreement verbs is from source to goal [...] \label{ex:amp1}
\ex The facing of the hand(s) is towards the object of the verb.\label{ex:amp2}

\end{xlist}
\end{exe}

The Agreement Morphology Principles account for both, regular and backwards verbs, which share the facing of the hands but differ in the direction of the path movement. According to (\ref{ex:amp1}), the direction of the path movement is controlled by the thematic roles source and goal which could be mapped on the arguments of \textsc{from} and \textsc{to} in Jackendoff's (1990) componential analysis. The facing of the hands, on the other hand, is controlled by the indirect object which is comparable with the dative object in spoken language. 

To account for the agreement facts of DGS and to model Meir’s principles in a constraint-based way, we 
develop a type-based representation of the existing classes of agreement verbs. In a first step  genuine agreement verbs are distinguished from plain verbs by stipulating two verbal subtypes, called \textit{plain\_verb} and  \textit{agr}(\textit{eement})\_\textit{verb} respectively. Secondly, the type \textit{agreement\_verb} is further  partitioned by two subtypes which are called \textit{reg}(\textit{ular})\_\textit{agreement\_verb} and \textit{back}(\textit{wards})\_\textit{agreement\_verb} in accordance with the analysis of Cormier et al. (1999). Additionally, there is a transitive and ditransitive variant of both subtypes. This is illustrated by the resulting signature in figure \ref{signature-ref}.




\begin{figure}[htbp]
\label{ex:signature}
\resizebox{\linewidth}{!}{
\resizebox{\textheight}{!}
%\resizebox{\textheight}{13pt}{
{\Tree
[.\textit{verb}
	\textit{plain\_verb}
	[.\textit{agr\_verb}
		[.\textit{reg\_agr\_verb}
		\textit{reg\_agr\_trans\_verb}
		\textit{reg\_agr\_ditrans\_verb}
		]
		[.\textit{back\_agr\_verb}
		\textit{back\_agr\_trans\_verb}
		\textit{back\_agr\_ditrans\_verb}
		]
	]
]}}
\caption{Partition of type \textit{verb}}\label{signature-ref}
\end{figure}




%\begin{figure}[htbp]
%\includegraphics[width=1.0\textwidth]{verbtypesforest.pdf}
%\caption{Partition of type \textit{verb}}caption{}\label{signature-ref}
%\end{figure}

With this type hierarchy at hand, we can now define appropriate lexical constraints that restrict verbal agreement with respect to a certain verbal class. 

Based on the usual HPSG practice to model agreement as structure-sharing between \textsc{index} values, we analyze syntactic agreement in DGS by manipulating the \textsc{orientation} value of \textsc{manual} and structure-sharing its \textsc{index} value with the \textsc{index} value of the indirect object on the \textsc{comps}-list of the respective verb. This accounts for Meir’s definition of syntactic agreement in (\ref{ex:amp2}), where the facing of the hands is syntactically controlled by the respective object in the argument structure. (\ref{ex:finger}) shows the partial description of the \textsc{phon} value that we assume for all verbs of type \textit{agreement\_verb} in the lexicon. The analysis is built on the analysis developed in Safar  \&  Marshall (2004).

\begin{exe}
\ex  \label{ex:finger}
\resizebox{\linewidth}{!}{\ms[agreement\_verb]{
	phon|manual & \ms{
		orient|finger|index & {\ibox{1}} locus\\
		movement & $\langle$\ms{
			begin|index & locus\\
			end|index & locus\\
		}$\rangle$\\
	}\\
	synsem|loc|cat|comps & $<$ ..., $\mathrm{NP}_{\ibox{1}}, ...>$ \\  
}}
\end{exe}

To implement Meir’s first clause of the Agreement Morphology Principles, which expresses her observation on thematic agreement, we add a relation of type \textit{transfer} to the \textsc{content} value for all verbs of type \textit{agreement\_verb}. This relation comes with three arguments: \textsc{source}, \textsc{goal} and \textsc{soa}. Our implementation of thematic agreement relies on the manipulation of the \textsc{movement} value: the \textsc{begin} value of \textsc{movement} is identified with the \textsc{source} value of the transfer relation and the \textsc{end} value with \textsc{goal} value of the same relation. This accounts for agreement as path movement in sign language.

\begin{exe}
\ex  \label{ex:agreev}
\ms[agreement\_verb]{
	phon|manual & \ms{
		% orient|finger|index  {\ibox{1}}\\
		movement  $\langle$\ms{
			begin|index & {\ibox{2}}\\
			end|index & {\ibox{1}}\\
		}$\rangle$\\
	}\\
	synsem|local & \ms{
		cont & \ms{relation & transfer\\
			source & {\ibox{2}}\\
			goal & {\ibox{1}}\\
			soa & qfpsoa}\\					
	}\\       
}
\end{exe}

Note that the semantics of any agreement verb is introduced by the \textsc{soa} value of the \textit{transfer} relation. This is necessary to prevent the prediction of a semantic hierarchy in which all semantic relations that are expressed by agreement verbs are at the same time subcases of a general transfer relation. Cognitively, this might be correct but in this paper, we do not argue for such a strong assumption and our analysis does not hinge on it.

The main difference between regular and backwards agreement verbs basically concerns the direction of the path movement which is mediated by the argument structural properties of the respective verbs. Following Meir’s insights on thematic agreement,  we assume that the path movement begins at the position of the subject (\textsc{source}) and ends at the position of the object (\textsc{goal}) in case of regular agreement verbs. By contrast, with backwards agreement verbs, path movement works the other way around. In this case, the movement starts at the position of the object (\textsc{goal}) and ends at the position of the subject (\textsc{source}). Again, this is realized as structure-sharing of positional \textsc{index} values as can be seen in (\ref{ex:regagreev}) and (\ref{ex:backagreev}) respectively.

\begin{exe}
\ex  \label{ex:regagreev}
\ms[regular\_agreement\_verb]{
	%phon|manual & \ms{
	%           orient|finger|index  {\ibox{1}}\\
	%           movement  $\langle$\ms{
	%           begin|index & {\ibox{2}}\\
	%           end|index & {\ibox{1}}\\
	%           }$\rangle$\\
	%           }\\
	synsem|local & \ms{cat & \ms{subj & $<${$\mathrm{NP}_{\ibox{2}}>$}\\
			comps & $<$...,$\mathrm{NP}_{\ibox{1}}, ...>$}\\ 
		cont & \ms{relation & transfer\\
			source & {\ibox{2}}\\
			goal & {\ibox{1}}\\
			soa & qfpsoa}\\					
	}\\       
}
\end{exe}

\begin{exe}
\ex  \label{ex:backagreev}
\ms[backwards\_agreement\_verb]{
	%phon|manual & \ms{
	%           orient|finger|index  {\ibox{1}}\\
	%           movement  $\langle$\ms{
	%           begin|index & {\ibox{1}}\\
	%           end|index & {\ibox{2}}\\
	%           }$\rangle$\\
	%           }\\
	synsem|local & \ms{cat & \ms{subj & $<${$\mathrm{NP}_{\ibox{2}}>$}\\
			comps & $<$...,$\mathrm{NP}_{\ibox{1}},...>$}\\ 
		cont & \ms{relation & transfer\\
			source & {\ibox{1}}\\
			goal & {\ibox{2}}\\
			soa & qfsoa}}\\       
}

\end{exe}

The only difference between the restrictions of both verb classes consists in the assignment of the indices. With regular agreement verbs the object is identified with the goal of the transfer relation, whereas with backwards agreement verbs, the subject is identified with the goal of the transfer relation. This is illustrated by the following structures. The full lexical specifications of the transitive regular agreement verb \textsc{help} and the ditransitive regular agreement verb \textsc{give} are depicted in (\ref{ex:transagrrev}) and (\ref{ex:ditransagrrev}). By contrast, (\ref{ex:inviteagree}) exemplifies the reverse specification of the transfer relation for the backwards agreement verb  \textsc{invite}.

\begin{exe}
\ex  \label{ex:transagrrev}
\ms[regular\_agreement\_trans\_verb]{
	phon|manual & \ms{
		orient|finger|index  {\ibox{1}}\\
		movement  $\langle$\ms{
			begin|index & {\ibox{2}}\\
			end|index & {\ibox{1}}\\
		}$\rangle$\\
	}\\
	synsem|local & \ms{cat & \ms{subj & $<\mathrm{NP}_{\ibox{2}}>$\\
			comps & $<\mathrm{NP}_{\ibox{1}}>$}\\ 
		cont & \ms{relation & transfer\\
			source & {\ibox{2}}\\
			goal & {\ibox{1}}\\
			soa & \ms[help\_rel]{
				helper & {\ibox{2}}\\
				helpee & {\ibox{1}}}}}\\       
}
\end{exe}

\begin{exe}
\ex  \label{ex:ditransagrrev}
\ms[regular\_agreement\_ditrans\_verb]{
	phon|manual & \ms{
		orient|finger|index  {\ibox{1}}\\
		movement  $\langle$\ms{
			begin|index & {\ibox{2}}\\
			end|index & {\ibox{1}}\\
		}$\rangle$\\
	}\\
	synsem|local & \ms{cat & \ms{subj & $<${$\mathrm{NP}_{\ibox{2}}>$}\\
			comps & $<$\_, $\mathrm{NP}_{\ibox{1}}>$}\\ 
		cont & \ms{relation & transfer\\
			source & {\ibox{2}}\\
			goal & {\ibox{1}}\\
			soa & \ms[give\_rel]{
				giver & {\ibox{2}}\\
				gift & {\ibox{3}}\\
				givee & {\ibox{1}}}}}\\       
}
\end{exe}

\begin{exe}
\ex  \label{ex:inviteagree}
\ms[backwards\_agreement\_trans\_verb]{
	phon|manual & \ms{
		orient|finger|index  {\ibox{1}}\\
		movement  $\langle$\ms{
			begin|index & {\ibox{1}}\\
			end|index & {\ibox{2}}\\
		}$\rangle$\\
	}\\
	synsem|local & \ms{cat & \ms{subj & $<\mathrm{NP}_{\ibox{2}}>$\\
			comps & $<\mathrm{NP}_{\ibox{1}}>$}\\ 
		cont & \ms{relation & transfer\\
			source & {\ibox{1}}\\
			goal & {\ibox{2}}\\
			soa & \ms[invite\_rel]{
				inviter & {\ibox{2}}\\
				invitee & {\ibox{1}}}}}\\ 
	
}

\end{exe}


\subsection{Plain verbs}
Recall that in the case of plain verbs such as \textsc{know} and \textsc{like}, phonological properties of the verb block the overt realization of agreement. This means that agreement with subject and object is prohibited because the beginning and endpoint of path movement and hand orientation are already lexically specified. As is illustrated in (\ref{ex:plain}) the respective \textsc{locus} values are instantiated by fixed values (i.e. lexically specified loci in the signing space) expressed by the variables \textit{i}, \textit{j} and \textit{k} representing indexical reference points.   


\begin{exe}
\ex  \label{ex:plain}
\ms[plain\_verb]{
	phon|manual & \ms{
		orient|finger|index|loc $\mathrm{\textit{k}}$\\
		movement   $\langle$\ms{
			begin|index|loc & i\\
			end|index|loc & j\\
		}$\rangle$\\
	}\\ 
	synsem|local|cat & \ms{head & verb\\
		subj & nelist\\
		comps & nelist}    
}
\end{exe}

As discussed above, sign languages have developed different means to overcome the agreement gap caused by plain verbs. DGS, for instance, makes use of the agreement auxiliary \textsc{pam}, which, just like regular agreement verbs, marks agreement manually by means of hand orientation and path movement. Therefore, \textsc{pam} insertion is a practicable option to express agreement overtly with plain verbs. Since the relevant phonological features \textsc{orientation} and \textsc{movement} are already lexically specified with plain verbs and hence not available for agreement inflection, \textsc{pam} can be used to agree with the subject and object of the plain verb and realize the corresponding features overtly.

In principle, there are different HPSG analyses available that have been proposed to account for several kinds of auxiliaries in spoken language and could be used to account for \textsc{pam}. One option is a lexical analysis of auxiliaries as proposed by Ackerman \& Webelhuth (1998). Following this account, \textsc{pam} would be added to the lexical entry of a plain verb. An alternative option would be that \textsc{pam} is subcategorized for a plain verb, and attracts all relevant arguments which are necessary to express agreement from this verb. This analysis accommodates the construction of verbal clusters in German by argument composition (cf. Hinrichs \& Nakazawa 1989, 1994; Müller 2007). It ensures that the agreement auxiliary \textsc{pam} may exploit path movement and hand orientation to express subject and object agreement. Hence, \textsc{pam} does not differ from regular agreement verbs in this respect. Nevertheless, \textsc{pam} acts as a syntactic marker only as it makes no use of the transfer relation as defined for regular and backwards agreement verbs.

 
%In addition, PAM can be productively used to extend the argument structure of the main verb as illustrated in (11) above. In this case, the main verb only selects one argument. The second (object argument) is introduced by PAM. Consequently, the argument structure extension triggers a corresponding transitive interpretation of the verb (i.e. ‘laught at’, ‘write to’ or ‘wait for’). The paper will give a full-fledged analysis of both constellations described using argument composition as proposed e.g. by Hinrichs/Nakazawa (1989). 

The partial description in (\ref{ex:personal}) gives the lexical specification of \textsc{pam}. It depicts that \textsc{pam} selects a verb of type \textit{plain\_verb} and attracts all arguments of this verb, which comprises the subject marked by tag [3] and the whole \textsc{comps} list marked by tag [4]. Since the indices of the plain verb's subject and object are structure shared with the beginning and the end point of the \textsc{movement} feature of \textsc{pam}, agreement is expressed purely syntactically. 


\begin{exe}
\ex  \label{ex:personal}
\resizebox{\linewidth}{!}{
\ms[personal\_agreement\_marker]{
	phon|man &\ms{
		orient|finger|index  {\ibox{2}}\\
		movement  $\langle$\ms{
			begin|index & {\ibox{1}}\\
			end|index & {\ibox{2}}\\
		}$\rangle$\\
	}\\
	ss|loc &\ms{cat & \ms{head &verb\\
			subj &{\ibox{3}}\\
			comps &{\ibox{4}} $\oplus\langle\mathrm{V}[$\textit{plain},
			{\textsc{subj}} {\ibox{3}}$<${$\mathrm{NP}_{\ibox{1}}>$},						   
			{\textsc{comps}} {\ibox{4}}$<$\ldots,$\mathrm{NP}_{\ibox{2}},$\ldots$>\mathrm{]}\rangle$}\\					
	}\\       
}}
\end{exe}

In order to account for the variation in the positioning of \textsc{pam} mentioned above, lexical precedence rules are needed that regulate whether \textsc{pam} has to be positioned pre- or postverbally. In cases where \textsc{pam} is used to extend the argument structure of the selected verb, an analysis is conceivable that adds an argument to the \textsc{comps} list of \textsc{pam} in dependence of a featural specification that marks that the corresponding main verb is one that qualifies in principle for argument structural extensions. Alternatively, one might argue that the \textsc{comps} list of \textsc{pam} is inherently specified for an object, which is then added to the \textsc{comps} list of a one-place main verb and triggers a corresponding transitive interpretation.


\section{Conclusion}

In sum, the HPSG analysis of agreement in DGS developed in this paper illustrates that a constraint-based lexical approach offers an elegant account of the modality-specific properties of sign language agreement. In particular, the interdependence of phonological, syntactic, and semantic properties of the verb and the simultaneous realization of agreement can be implemented in a straightforward way using cross-modular constraints on syntactic and thematic agreement in DGS. Moreover, we can account for agreement in DGS without assuming additional morphosyntactic features or specific agreement morphemes since the agreement principles directly operate on phonological locus features of the verb and its arguments. Finally, our analysis correctly predicts the distribution of the agreement auxiliary \textsc{pam} in DGS.

\nocite{*}
\bibliographystyle{unified} 
\bibliography{slagreement.bib}


%
%\section{Input Brendan Ryan}
%
%\begin{exe}
%\ex  \label{ex:neigbor}
%\begin{xlist}
%	\ex \textsc{conference person++ be-present} \\
%	‘There were many people present at the conference.'
%	
%	\ex \textsc{mother ix\textsubscript{3a} neigbor new ix\textsubscript{3b} like \textsubscript{3a}pam\textsubscript{3b}}\\
% ‘(My) mother likes the new neighbor.'
%\end{xlist}
%\end{exe}
%
%

%\section{Attribute-Value Matrix/Merkmalstruktur}
%
%%\mathrm wurde überall enfernt
%
%\ms[typ]{
%	merkmal & wert$_{\ibox{1}}$\\
%	merkmal2 & \ms{
%		merkmal3|merkmal4 & \ibox{1}wert3\\
%	}\\
%}
%
%




\end{document}
