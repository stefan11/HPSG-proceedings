%%%%%%%%%%%%%%%%%%%%%%%%%%%%%%%%%%%%%%%%%%%%%%%%%%%%%%%%%
%%   $RCSfile: example.tex,v $
%%  $Revision: 1.1 $
%%      $Date: 2004/08/11 09:52:43 $
%%     Author: Stefan Mueller (CL Uni Bremen)
%%    Purpose: Example file for contributions to the HPSG proceedings
%%   Language: LaTeX
%%%%%%%%%%%%%%%%%%%%%%%%%%%%%%%%%%%%%%%%%%%%%%%%%%%%%%%%%


\documentclass[11pt,a4paper,fleqn]{article}
\usepackage{times}
\pagestyle{empty}

\usepackage[sectionbib]{natbib}


% If you want to make your paper more usable for readers,
% insert the following code.
% It links references to sections, examples, and citations
% in the document and provides bookmarks for sections and subsections.

%% \usepackage[bookmarks=true,bookmarksopen=true,hyperindex=true,breaklinks=true,draft=false,plainpages=false,hyperfootnotes=false,%
%% colorlinks=false, pdfborder={0 0 0},%
%% %pdftex=true]{hyperref}  % use this if you use pdflatex
%% ps2pdf=true]{hyperref}% use this if you use ps2pdf, i.e. if you use tree-dvips




\title{My Really Interesting Discovery}

\author{Stefan M�ller\\
Univerit�t Bremen}

\begin{document}
%\maketitle

\begin{center}
\Large
My Really Interesting Discovery\\[\baselineskip]

Author's Name\\[\baselineskip]
Author's Affiliaton\\[3\baselineskip]

                Proceedings of the HPSG04 Conference\\
                Workshop on Semantics in Grammar Engineering\\[\baselineskip]

                Center for Computational Linguistics\\
                Katholieke Universiteit Leuven\\[\baselineskip]

                        Stefan M{\"u}ller (Editor)\\[\baselineskip]

                                2004\\[\baselineskip]

                          CSLI Publications\\[\baselineskip]

              http://csli-publications.stanford.edu/

\end{center}

\newpage

\begin{abstract}
I made a really important discovery that will change linguistics.
\end{abstract}

\setcounter{footnote}{2}
\renewcommand{\thefootnote}{\fnsymbol{footnote}}
\footnotetext{
I thank X and Y.}
\renewcommand{\thefootnote}{\arabic{footnote}}
\setcounter{footnote}{0}


\section{Introduction}


\section{Conclusion}

\citet{ps,ps2} were right.

\bibliographystyle{natbib.fullname} 
\bibliography{biblio}



\end{document}

% Local variables:
% mode: font-lock
% End:

