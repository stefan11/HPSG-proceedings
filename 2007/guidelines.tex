%%%%%%%%%%%%%%%%%%%%%%%%%%%%%%%%%%%%%%%%%%%%%%%%%%%%%%%%%
%%   $RCSfile: example.tex,v $
%%  $Revision: 1.1 $
%%      $Date: 2004/08/11 09:52:43 $
%%     Author: Stefan Mueller (CL Uni Bremen)
%%    Purpose: Example file for contributions to the HPSG proceedings
%%   Language: LaTeX
%%%%%%%%%%%%%%%%%%%%%%%%%%%%%%%%%%%%%%%%%%%%%%%%%%%%%%%%%


\documentclass[11pt,a4paper,fleqn]{article}


\usepackage{times,8bit,float}


\usepackage[sectionbib]{natbib}

\usepackage{german}
\selectlanguage{USenglish}

% If you want to make your paper more usable for readers,
% insert the following code.
% It links references to sections, examples, and citations
% in the document and provides bookmarks for sections and subsections.

\usepackage[bookmarks=true,bookmarksopen=true,hyperindex=true,breaklinks=true,draft=false,plainpages=false,hyperfootnotes=false,%
colorlinks=false, pdfborder={0 0 0},%
pdftex=true]{hyperref}  % use this if you use pdflatex
%ps2pdf=true]{hyperref}% use this if you use ps2pdf, i.e. if you use tree-dvips




\title{Guidelines for Type Setting Your Contribution to the HPSG Proceedings}

\author{Stefan M�ller\\
Free Univerity Berlin}

\begin{document}
\maketitle


\section{Things to Submit}

Please submit:
\begin{enumerate}
\item your paper in pdf format\footnote{
As with last year, we will produce a large pdf file that includes all
of the papers.  This way, readers can download the entire proceedings
as a book.

If you absolutely cannot do PDF, please let me know as soon as
possible and I will accomodate you in some way: point you to some
reasonable converters, help you with the conversion, etc.}
%
 named `kruger.pdf', if your lastname is \emph{Kruger}.



\item an abstract
\end{enumerate}
The abstract will be presented at a separate HTML page. You may submit it in ASCII [please remove \LaTeX{} mark up, thanks]
or---if you have special formatting---in HTML. 
To get an idea about how this looks like,
you may look at the proceedings of previous years at:\newline
\url{http://cslipublications.stanford.edu/HPSG/}

\section{Deadline}

The deadline for submission is October 15th.

\section{Page Limit}

The page limit for full papers is 20 pages, title page not included.

The page limit for posters is 15 pages, title page not included.

\section{Title Page}


The title page is not included in the 20/15 pages.

Make sure this is on a separate page with all lines centered:

\begin{center}
\Large
Title of the Paper\\[\baselineskip]

Author's Name\\[\baselineskip]
Author's Affiliaton\\[3\baselineskip]

                Proceedings of the HPSG07 Conference\\[\baselineskip]

Stanford Department of Linguistics and CSLI's LinGO Lab\\[\baselineskip]

                        Stefan M{\"u}ller (Editor)\\[\baselineskip]

                                2007\\[\baselineskip]

                          CSLI Publications\\[\baselineskip]

              http://csli-publications.stanford.edu/

\end{center}

\section{Abstract}

We encourage the inclusion of a short (one paragraph)  abstract just after the title page.
(no pagebreak before the main text, please)


\section{Details on Formatting}

\subsection{\LaTeX{} Users}

\subsubsection{Styles}

Please adapt the \verb+example.tex+ file at:\newline
\url{http://cslipublications.stanford.edu/HPSG/7/call.shtml}\newline
and use the BibTeX file provided there.

\subsubsection{PDF Creation}

If you do not fancy stuff that needs PostScript you may produce
the pdf file by using \verb+pdflatex+. If you use \verb+pstricks+ or
similar packages, please use \verb+dvips+ and \verb+ps2pdf13+.

Using \verb+ps2pdf13+ helps avoiding problems with ligaturs like in `fi'
or `ff'. If you do not have this on your machine, using \verb+ps2pdf+ is okay.

Please use the \verb+ps2pdf+-option \verb+-dAutoRotatePages=/None+  to avoid auto rotation
of figures.

If you are using \verb+dvips+, please use the option \verb+-t a4+. This tells
\verb+dvips+ to use A4 paper.


\subsubsection{Huge AVMs and Trees}
\label{huge}

Please typeset trees, avms, and formulae in a way that they fit the
\verb+\textwidth+. If you see no other way to reduce the size of the respective
objects, you may use \verb+resizebox+ from the package \verb+graphicx+.
{\small
\begin{verbatim}
\resizebox{\linewidth}{!}{%
\ms{
 synsem & \onems{ loc$|$cat$|$subcat del(\ibox{2},\ibox{1})\\
                }\\
 head-dtr & \onems{ synsem$|$loc$|$cat$|$subcat \ibox{1} \\
                  }\\
 non-head-dtrs & \sliste{\onems{ synsem \ibox{2}\\ 
                               }}\\[2mm]
}}
\end{verbatim}
}

If you use \verb+graphicx+ you have to produce the pdf file from
an intermediate PostScript file.

\subsubsection{Hyphenation}
\label{hyphenation}

If you write things like \verb+head-driven+ or very long pathes like
{\sc snysem$|$loc$|$cat$|$head$|$mod$|$loc}, \LaTeX{} does not do hyphenation
(in the part following the dash).

If you use \verb+german.sty+ you get additional markup that allows for proper hyphenation:
\begin{verbatim}
head"=driven

{\sc snysem$|$""loc$|$""cat$|$""head$|$""mod$|$""loc}
\end{verbatim}
With this markup even long pathes like {\sc snysem$|$loc$|$cat$|$head$|$mod$|$loc$|$cat$|$""head}
are typeset properly. Alternatively you my write
\begin{verbatim}
{\sc snysem$|$\-loc$|$\-cat$|$\-head$|$\-mod}
\end{verbatim}
which introduces a dash at the place of the linebreak:
{\sc snysem$|$\-loc$|$\-cat$|$\-head$|$\-mod$|$\-loc$|$\-cat$|$\-head}.

If you use \verb+german.sty+ do not forget to declare English as the language
you are using:
\begin{verbatim}
\selectlanguage{USenglish}
\end{verbatim}
Otherwise the section name for references comes out in German.

\subsection{Others (StarOffice, Word, \ldots)}

\subsubsection{Font}

Please use a 11pt times (Type 1 font). The title page should be set with 18pt times (Type 1 font).

\subsubsection{No Page Numbers}

Please submit your paper without page numbering.  I will add in
the page numbering on the PDF file as I am putting the
proceedings together.

\subsubsection{Justification}

Please make sure that the text is typeset in justification,
i.e. with the text aligned at both the left-hand side and the right-hand side.

\subsubsection{Papersize and Margins}
\label{margins}

Please use A4 paper:
\begin{table}[H]
\begin{tabular}{@{}ll}
paperheight  = 297mm\\
paperwidth   = 210mm
\end{tabular}
\end{table}
%
The text size should be:
\begin{table}[H]
\begin{tabular}{@{}ll}
textwidth  = 120mm\\
textheight = 201mm
\end{tabular}
\end{table}

\begin{table}[H]
\begin{tabular}{@{}l@{ = }l}
distance from the top & 50 mm\\
distance from the left-hand side & 45 mm\\
distance from the right-hand side & 44 mm\\
distance from the bottom & 48 mm\\
\end{tabular}
\end{table}

This is important since otherwise automatic page numbering will not work.
If you are not sure whether you did things right, you may compare your file
with (my) papers from the past years:\newline 
\url{http://cslipublications.stanford.edu/HPSG/5/toc.shtml#stmue}

\section{Things You Should Not Do}

\begin{itemize}
\item Do not change the margins or other \LaTeX{} internal values! Page numbers will be inserted automatically and
      if you changed things this will not work and cause me and you additional work.
\item Do not change the font size.
\end{itemize}


\section{Checklist}

\begin{itemize}
\item Do you have the right margins (\LaTeX{} $\to$ no problem if you didn't change
      the styles, others see Section~\ref{margins})?
\item Are you within the page limit of 20 pages + title page?
\item Did you suppress the page numbers? (including the number on the title page)
\item If you look at figures in the pdf file, do rotated figures flip back?
      Please edit your text in a way that figures do not flip back, since this
      breaks the process of automatic page numbering.

If you used \LaTeX{} to produce your document you can avoid rotating figures by calling
\verb+ps2pdf13+ with the option \verb+-dAutoRotatePages=/None+.
\item Is there something sticking out at the right-hand side of the text?
      If so, see Section~\ref{hyphenation}.
\item Are there coloured URLs or something that like in your paper? If so,
      please change this to the text colour.
\item Do you have AVMs, equations, or trees that do not fit the textwidth?
      If so, please fix this. See Section~\ref{huge}.
\item Is the name of the pdf file `author.pdf', where `author' is your lastname or a list of the lastnames of several authors
      separated by `-'?
\item Does the paper contain an abstract?
\item Do you have an abstract to send separately?
\item Did you remove all \LaTeX{} markup from the abstract? (Yes, `\verb+---+' is also markup)
\end{itemize}
%
Great! I am looking forward to your submission!
Please send them to\newline \href{mailto:Stefan.Mueller@fu-berlin.de}{Stefan.Mueller@fu-berlin.de}.



If you have any questions, please contact me at the given adress.



%% \bibliographystyle{natbib.fullname} 
%% \bibliography{biblio}



\end{document}

% Local variables:
% mode: font-lock
% End:

